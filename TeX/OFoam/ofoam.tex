\chapter{OpenFOAM case definition}
    Case in \oFoam is defined in a very segregated way. Each piece of the case is defined in a separate text file placed in the specific directory in case directory tree (see figure \ref{fig::ofoam_folders}). 
    The most important thing to remember while editing those files is that they are essentially C++ code.
    Which means that standard C++ programming rules apply. For example each line must be ended with a semicolon. This section will describe in detail how to set up a full \oFoam case starting from mesh generation and finishing on post-processing. Input files for both cases featured in this work can be viewed and downloaded from~\cite{git}.

    \begin{figure}[h]
      
    \dirtree{%
            .1 case\DTcomment{main directory of the case}.
            .2 constant\DTcomment{constant properties of the case e.g. mesh description}.
            .2 system\DTcomment{numerical schemes, solvers and simulation settings}.
            .2 0\DTcomment{initial conditions for all fields including boundary conditions}.
            .2 100\DTcomment{solution for timestep $t=100$}.
            .2 processor*\DTcomment{part of the case that was decomposed for core "\texttt{*}"}.
    }
    \caption{Typical \oFoam case folder structure}
    \label{fig::ofoam_folders}
    \end{figure}

    \section{Turbulent flow over flat plate}
        First step in defining the case is selecting properties of transported fluid.
        This can be done by creating appropriate file in \texttt{constant} folder:
        \begin{lstlisting}[language=bash]
            mkdir constant
            touch constant/transportProperties
        \end{lstlisting}
        and editing it so it looks like this:
        \lstinputlisting[language=C++,firstline=8,lastline=40,basicstyle=\small\ttfamily]{../final_cases/NASA_fp/constant/transportProperties}

        Code starting with \lstinline[language=C++]{FoamFile} is a definition of chosen file.
        It has file's location specified and it's name. Most of the files created will be instances of \lstinline[language=C++]{dictionary} class and will have the same version and format.

        Below transport model is declared. 
        Simulated fluid has Newtonian constitutive relation.
        Next fluid kinematic viscosity is defined.
        Value $\num{1.5e-5}$ selected for this simulation is appended by description of its physical units (which is optional).
        Seven fields in braces corresponds to powers of specific SI units:
        \begin{lstlisting}[language=C++]
            [kg m s K mol A cd]
        \end{lstlisting}
        Kinematic viscosity has a unit of \si{\meter\squared\per\second} which yields \lstinline[language=C++]{[ 0 2 -1 0 0 0 0 ]} as \oFoam unit description.

        Since simulated flow is turbulent in nature, solver also needs information about turbulent modelling technique and chosen turbulence model.
        Those settings are in file named \texttt{turbulenceProperties} placed in the same folder.
        Contents of this file should look like this:
        \lstinputlisting[language=C++,firstline=8,lastline=26,basicstyle=\small\ttfamily]{../final_cases/NASA_fp/constant/turbulenceProperties}
        First \lstinline{simulationType} field must be defined. Here \lstinline{RAS} is \oFoam keyword for RANS simulation. 
        Next \lstinline{RAS} dictionary must be defined which includes choosing turbulence model, switching on the turbulence and additional options.

        \subsection{Mesh generation}
            Next step is mesh generation using \lstinline{blockMesh} utility. It is capable generating structural hexagonal meshes in three dimensions. 
            Input for \lstinline{blockMesh} is also defined in text file named \texttt{blockMeshDict} and placed in \texttt{case/system} directory. Detailed explanation and description of \lstinline{blockMesh} format is available here~\cite{blockMesh}.

            For this case \texttt{blockMeshDict} should look like this:
            \lstinputlisting[firstline=8,language=C++,basicstyle=\small\ttfamily]{../final_cases/NASA_fp/system/blockMeshDict}

            The last boundary patch is of type \lstinline{empty}. Since \oFoam can only simulate three dimensional flows. Empty boundary condition is used to reduce case to two dimensional. It should be specified in two dimensional cases on faces perpendicular to $z$ axis.

        \subsection{Defining boundary and initial conditions}
            Initial and boundary conditions must be specified for each field that is calculated during run time: $\bVec{u}$, $p$, $k$, $\omega$ and $\nu_T$.
            Every simulation in \oFoam is iterated in time (even steady ones; time is than just iteration number).
            Definitions for each of those fields is specified in the \texttt{0} folder (see figure \ref{fig::ofoam_0}) which can be interpreted as flow data from time $0$.
            \begin{figure}[b!]
      
                \dirtree{%
                        .1 case.
                        .2 0\DTcomment{initial time folder}.
                        .3 U\DTcomment{velocity field}.
                        .3 p\DTcomment{pressure field}.
                        .3 k\DTcomment{turbulent kinetic energy field}.
                        .3 omega\DTcomment{turbuelnt dissipation rate}.
                        .3 nut\DTcomment{turbulent visocsity}.
                }
                \caption{File structures defining initial and boundary conditions}
                \label{fig::ofoam_0}
            \end{figure}
            Each of the files has the same structure (and files in directories describing further times e.g. \texttt{100/U} will have the same structure as well) that can be divided into 4 sections:
            \begin{description}
                \item[\texttt{FoamFile}] - standard file description. Class of each field is defined here: \lstinline{volVectorField} is a vector field e.g. $\bVec{u}$ and \lstinline{volScalarField} is a typical scalar quantity, for example $p$.
                \item[\texttt{dimension}] - definition of physical units of the quantity.
                \item[\texttt{internalField}] - values of field variables inside the domain. In the initial time step it is connivent to use \lstinline{uniform} keyword to set one value for each cell, for example \lstinline[language=C++]{internalField   uniform (75 0 0);} initialises velocity vector with $x$ component equal to $\num{75}$.
                \item[\texttt{boundaryField}] - definitions of boundary conditions. For each \lstinline{patch} defined in \texttt{blockMeshDict} user must specify boundary condition type and value (if needed). Several of possible types are:
                \begin{description}
                    \item[\texttt{fixedValue}] - condition of Dirichlet type imposing uniform value;
                    \item[\texttt{zeroGradient}] - zero Neumann boundary condition;
                    \item[\texttt{symmetryPlane}] - condition imposing symmetry;
                    \item[\texttt{empty}] - condition placed in two dimensional cases on unnecessary patches.
                \end{description} 
            \end{description}

            
            
            Exemplary text file for velocity field is shown below:
            \lstinputlisting[language=C++, firstline=8,basicstyle=\small\ttfamily]{../final_cases/NASA_fp/0/U}
        
        \subsection{Setting up the simulation}
            In order to run any simulation a file with solver setting must be specified.
            It stores information about time step, iteration numbers and other options.
            File must be named \texttt{controlDict} and must be placed in \texttt{case/system} directory.
            
            Again it has \lstinline[language=C++]{FoamFile} field in the begging. Than it is appended by following entries:
            \begin{description}
                \item[\textbf{application}] - in this place numerical solver is specified, in this simulations \lstinline[language=C++]{simpleFoam} was chosen. It is a solver for steady turbulent problems.
                \item[\textbf{iterations}] - specification of iterations or time stepping (as here they are interchangeable) is rather self explanatory and is achieved in the following way:
                    \begin{lstlisting}[language=C++]
startFrom       startTime;
startTime       41000;
stopAt          endTime;
endTime         50000;
deltaT          1;
                    \end{lstlisting}
                \item[\textbf{saving solutions}] - there are several options for saving solutions while the simulation is running. Again they are mostly self explanatory:
                \begin{lstlisting}[language=C++]
writeControl    timeStep;
writeInterval   2000;
purgeWrite      0;
                \end{lstlisting}
                The last entry in the above list describes when solutions from previous time steps should be deleted (integer specified in \lstinline[language=C++]{purgeRight} field describes maximal number of time directories; \lstinline[language=C++]{0} means no overwriting, each solution is saved).
                \item[\textbf{I/O options}] - fields describing writing precisions for fields and time:
                \begin{lstlisting}[language=C++]
writeFormat     ascii;
writePrecision  8;
writeCompression off;                    
timeFormat      general;                 
timePrecision   8;
runTimeModifiable true;
                \end{lstlisting}
                Last entry in the above list enables altering case files during simulation time i.e.\ changing of postprocessing settings, time step etc.
            \end{description}
            In depth description of options available in \texttt{controlDict} can be found in~\cite{controlDict}.
        \subsection{Selecting numerical schemes}
            Each differential operator is in any equation must be simplified into its discrete form in order to solve such problem. Description of numerical schemes used to represent those operators is in the file \texttt{case/system/fvSchemes}.

            Another reason behind \oFoam being noticeably harder to use than other software packages that are available because is, it does not specify any standard numerical schemes for a chosen problem.
            User must specify a discrete scheme for each differential operator manually.
            For example time derivative can be defined in the following way:
            \begin{lstlisting}[language=C++]
ddtSchemes
{
    default steadyState; //default scheme (used for every field variable)
    ddt(k)  backward;    //scheme for k equation (overrides default scheme)
}
            \end{lstlisting}
            If simulation requires calculation of wall distance $y$ that method should also be specified here:
            \begin{lstlisting}[language=C++]
wallDist
{
    method          meshWave;
}
            \end{lstlisting}
            Additional documentation about numerical schemes can be found here~\cite{fvSchemes}.
        \subsection{Setting up the linear solvers}
            After the discretisation each equation is represented by system of linear equations, that is solved iteratively. User can specify a solver type and its parameters inside \texttt{case/system/fvSolution}. Since \texttt{simpleFoam} is a segregated solver (it solves for each component of $\bm{u}$ and $p$ separately) user must specify one solver for components velocity and the other one for pressure. Another two solvers for turbulent variables are also necessary. For example settings for $p$ may look like this:
            \begin{lstlisting}[language=C++]
p
{
    solver          GAMG;           //type of the solver
    tolerance       1e-16;          //final tolerance of the residual
    relTol          0.001;          //relative tolerance of the residual
    maxIter         1000;           //maximum number of iterations
    smoother        GaussSeidel;    //type of smoother
    nPreSweeps      1;              //-----------------------------
    nPostSweeps     3;              //
    nFinestSweeps   1;              //
    cacheAgglomeration on;          //settings for GAMG solver
    agglomerator    faceAreaPair;   //
    nCellsInCoarsestLevel 128;      //
    mergeLevels     1;              //-----------------------------
}
            \end{lstlisting}
            Fields \lstinline[language=C++]{tolearnce}, \lstinline[language=C++]{relTol} and \lstinline[language=C++]{maxIter} are stop criteria - when solver reaches any of those it will stop iterating. The other fields are options for chosen solver.
            
            In the same file settings for SIMPLE algorithm implemented in \texttt{simpleFoam} are placed:
            \begin{lstlisting}[language=C++]
SIMPLE
{
    nNonOrthogonalCorrectors 0; //specifies repeated solutions of the 
                                //pressure equation, used to update the
                                //explicit non-orthogonal correction

    residualControl             //stopping criteria for the algorithm 
    {                           //based on residual
        p               1e-8;
        U               1e-8;
        "(k|epsilon|omega)" 1e-8; //one criterion for multiple fields
    }
}
            \end{lstlisting}
            One setting or solver can be applied on all the fields at once, by using parenthesis and ''or'' operator (''\texttt{|}'') as seen in above example. 
            
            Another very important simulation parameter is also specified here - under-relaxation coefficients. User can specify a value for each of the variables:
            \begin{lstlisting}[language=C++]
relaxationFactors
{
    equations
    {
        U               0.5;
        p               0.3;
        k               0.5;
        omega           0.5;
    }
}
            \end{lstlisting}
            More in depth documentation of \texttt{fvSolution} file can be found here~\cite{fvSolution}.
        \subsection{Mesh decomposition}
            As in most CFD codes in \oFoam mesh can be decomposed into many parts in order to enable palatalisation of necessary calculations.
            It is achieved by creating file named \texttt{decomposeParDict} in \texttt{system} directory and running \texttt{decomposePar} utility. Insides of this file describes how mesh should be divided:
            \lstinputlisting[language=C++, firstline=8,basicstyle=\small\ttfamily]{../final_cases/NASA_fp/system/decomposeParDict}
            After parallel calculations are completed mesh and saved results can be recombined to original state by calling \texttt{reconstructPar}
        \subsection{Running the solver}
            After mesh is generated, initial and boundary conditions imposed and simulation settings set up, a solver can be run. It is achieved by using \texttt{simpleFoam} command. It will run SIMPLE segregated solver along with appropriate solvers for chosen turbulence models in serial mode (on one computation core).

            Calculations can be run in parallel mode using MPI. In order to do that first host file must be created. In case of running solver on one computer with 2 cores it should look like this:
            \lstinputlisting{../final_cases/NASA_fp/machines}
            Than solver is called with following command:
            \begin{lstlisting}
mpirun --hostfile <path to host file> -np 2 simpleFoam -parallel 
            \end{lstlisting}
            It is often useful to redirect output from \texttt{simpleFoam} to log file and run it in the background by appending above command with:
            \begin{lstlisting}
> <name of log file> & 
            \end{lstlisting}
        \subsection{Understanding and plotting the residuals}
            Example output from \texttt{simpleFoam} solver can look in the following way:
            \begin{lstlisting}[basicstyle=\footnotesize\ttfamily]
Time = 5006

smoothSolver:
Solving for Ux, Initial residual = 2.539e-05, Final residual = 1.787e-07, No Iterations 4
smoothSolver:  
Solving for Uy, Initial residual = 3.608e-05, Final residual = 2.802e-07, No Iterations 3
GAMG:
Solving for p, Initial residual = 5.525e-05, Final residual = 5.510e-10, No Iterations 586
time step continuity errors: 
sum local = 2.544e-09, global = -1.575e-10, cumulative = -6.343e-10
DILUPBiCGStab:
Solving for omega, Initial residual = 3.645e-09, Final residual = 9.800e-17, No Iterations 3
smoothSolver:
Solving for k, Initial residual = 1.321e-05, Final residual = 7.552e-17, No Iterations 21
ExecutionTime = 11.74 s  ClockTime = 12 s             
            \end{lstlisting}

            Algorithm outputs two residual values for each variable:
            \begin{description}
                \item[Initial residual] - measure of how well old solution fits new system of algebraic equations.
                \item[Final residual] - measure of convergence of linear solver. 
            \end{description}
            The first one is, at the same time, a measure of convergence of solution of nonlinear partial differential equation.
            
            In order to monitor solution convergence, plot of initial residuals against number of iterations is often useful. It can be achieved by using \texttt{gnuplot}:
            \begin{lstlisting}
gnuplot residuals
            \end{lstlisting}
            File \texttt{residuals} is a bash script\footnote{Courtesy of user \textit{wolle1982} of \url{www.cfd-online.com} forum~\cite{residuals}.} reading residual data from file named \texttt{log}:
            \begin{lstlisting}[showstringspaces=false,language=bash, basicstyle=\footnotesize\ttfamily]
set logscale y
set title "Residuals"
set ylabel 'Residual'
set xlabel 'Iteration'
plot "< cat log | grep 'Solving for Ux' | cut -d' ' -f9 | tr -d ','" title 'Ux' with lines,\
"< cat log | grep 'Solving for Uy' | cut -d' ' -f9 | tr -d ','" title 'Uy' with lines,\
"< cat log | grep 'Solving for Uz' | cut -d' ' -f9 | tr -d ','" title 'Uz' with lines,\
"< cat log | grep 'Solving for omega' | cut -d' ' -f9 | tr -d ','" title 'omega' with lines,\
"< cat log | grep 'Solving for k' | cut -d' ' -f9 | tr -d ','" title 'k' with lines,\
"< cat log | grep 'Solving for p' | cut -d' ' -f9 | tr -d ','" title 'p' with lines
pause 1
reread
            \end{lstlisting}
            It can be called while the simulation is running in the background and solver's output is redirected to a file.
        \subsection{Postprocessing}
            \oFoam has many postprocessing tools built in e.g.\ an utility used to calculate $y^+$ value in cells adjacent to viscous walls.
            Postprocessing routines can be used in three ways:
            \begin{enumerate}
                \item after the simulation - using \texttt{postProcess} command
                \item after the simulation - running chosen solver with \texttt{-postProcess} option
                \item during run time
            \end{enumerate}
            The first option is the simplest. In order to calculate any field variable, for example Courant number or $y^+$ user must call:
            \begin{lstlisting}[language=bash]
postProcess -func <field name> 
            \end{lstlisting}
            List of fields that can be calculated can be found by typing:
            \begin{lstlisting}[language=bash]
postProcess -list
            \end{lstlisting}
            For vorticity field, appropriate command will look like this:
            \begin{lstlisting}[language=bash]
postProcess -func vorticity
            \end{lstlisting}
            Program that is executed will calculate values of chosen field in each time step directory.

            Another option is to include postprocessing tools inside \texttt{controlDict} file. It is useful in situations when postprocessing requires some additional code i.e. turbulence model as in the case of wall shear stress calculation.
            In order to do that user can append \texttt{controlDict} with the following code:
            \begin{lstlisting}[language=C++]
functions
{
    #includeFunc "wallShearStress"
}
            \end{lstlisting}
            Any function that is listed here will be evaluated during run time (evaluation time is selected by \texttt{writeControl} option) and its outcome will be saved in appropriate time step directory.
            Functions can be also evaluated after simulation using \texttt{<chosen solver> -postProcess} command (more information about postprocessing and available utilities can be found here~\cite{postProc}).
        
            Postprocessing can also be done in third party software \textbf{ParaView}. It is shipped with most of \oFoam distributions and can be lunched by calling \texttt{paraFoam} utility inside case directory.
        
        \subsection{Sampling simulation data}
            Various kinds of data, for example components of velocity field, can be extracted along specified line or surface. It is achieved by introducing a new line in \lstinline[language=C++]|functions{...}| block in \texttt{controlDict} file:
            \begin{lstlisting}[language=C++]
#includeFunc "sample"
            \end{lstlisting}
            Next new file \texttt{case/system/sample} must be created with definitions of sampling surfaces. There are many options available. In case of this two dimensional simulation only extracting data along line is necessary (as seen in figure \ref{fig::flatplate}). Example \texttt{sample} file can look like this:
            \begin{lstlisting}[language=C++]
libs            ("libsampling.so"); //inclusion of executable library
type            sets;               //sets means sampling along lines
writeControl    writeTime;          
interpolationScheme cellPoint;      //selection of interpolation scheme
setFormat       raw;                //output data format
sets                                //set definitions
(
    plateHorizontal           //first set, defined as a line  
    {                         //to end with 1000 points
        type    face;
        axis    xyz;                //specifying what coordinates 
                                    //should be outputted
        start   (0.1 0.2 0);
        end     (2 0.4 0);
        nPoints 1000;
    }

    plateBegin                      //second set defined as cloud of points
    {
        type    cloud;
        axis    xyz;
        points ( #include "points" ); //inclusion of file 
                                      //with points definitions
    }
);

fields          (p U k omega);      //selection of fields to be sampled
            \end{lstlisting}
            As dictionary files are essentially a code in C++ when non uniform spacing is needed a claud of points can be simply ''pasted'' into \texttt{sample} file using standard \lstinline[language=C++]{#include} preprocessor directive.
            Data in \texttt{case/system/points} files should has following format:
            \begin{lstlisting}[language=C++]
(x1 y1 z1)  //standard openFOAM vectors, not separated by any delimiter
(x2 y2 z2)
            \end{lstlisting}
            More in detail description of sampling abilities of \oFoam can be found~\cite{sampl}
    \section{Trimmed domain - case setup}
        Trimmed domain has four distinct boundaries (see figure \ref{fig::flatplate}):
        \begin{description}
            \item[A-B]: horizontal inlet boundary that is situated \SI{0,1}{\metre} from the start of the plate.
                Velocity profile should vary across this boundary.
            \item[B-C]: skew boundary above plate surface. Velocity field is also not uniform on this boundary.
            \item[C-D]: domain outlet.
            \item[D-A]: plate surface modelled as viscous wall.   
        \end{description}
        
        Boundary conditions on edges \textbf{C-D} and \textbf{D-C} are straight forward and set up in the same way as in previous case. In order to set appropriate velocity profiles on rest of boundaries \texttt{timeVaryingMappedFixedValue} type of boundary condition has been used.
        \subsection{Velocity and turbulent fields BCs - \texttt{timeVaryingMappedFixedValue}}
            In order to use \texttt{timeVaryingMappedFixedValue} type, first correct boundary condition should be defined in \texttt{0/U} file in following manner:
            \begin{lstlisting}[language=C++]
inlet
{
    type            timeVaryingMappedFixedValue;
    offset          (0 0 0);    //offset for three dimensional vector field
    setAverage      off;        //scale field to have an certain 
                                //average value
}
            \end{lstlisting}
            As name suggests this type of boundary condition lets user to specify fields that are both changing in time and space.
            It is achieved by creating a directory named the same as chosen patch (in this case it was \texttt{inlet}) in \texttt{constant/boundaryData} directory.
            Files inside this folder have similar structure as definitions of boundary conditions in \texttt{case} directory (see figure \ref{fig::ofoam_timeVar}). For each chosen time step patch field values must be defined along with interpolation points (that are time invariant). If only initial time is specified boundary conditions will not change in time.
            \begin{figure}[ht!]
      
                \dirtree{%
                        .1 case.
                        .2 constant\DTcomment{constants folder}.
                        .3 boundaryData\DTcomment{data for \texttt{timeVaryingMappedFixedValue} boundary condition}.
                        .4 inlet\DTcomment{file with space varying data for inlet patch}.
                        .5 points.\DTcomment{file describing points where sampling was made}.
                        .5 0\DTcomment{initial time step}.
                        .6 U\DTcomment{velocity field}.
                        .6 k\DTcomment{turbulent kinetic energy field}.
                        .6 omega\DTcomment{turbulent dissipation rate}.
                        .5 <\textit{some other time step}>.       
                        .4 <\textit{name of other patch}>.             
                }
                \caption{File structures defining \texttt{timeVaryingMappedFixedValue} conditions}
                \label{fig::ofoam_timeVar}
            \end{figure}

            In order to define points and values of fields in those points sampling along lines were used.
            In this case flow is two dimensional. Having in mind that \oFoam is a three dimensional solver, it is better to provide tho sets of points for each patch - laying on both of \texttt{empty} planes\footnote{It can be achieved for example by copying and processing sampled data in a simple Matlab script.}. It will ensure that values are interpolated correctly across the "empty" dimension. File with points definition should look in the following way:
            \begin{lstlisting}[language=C++]
4 //number of points
(
(1.0000000000e-01 0.0000000000e+00 0.0000000000e+00) //coordinates
(1.0000000000e-01 2.0000000000e-06 0.0000000000e+00)
(1.0000000000e-01 4.4430910000e-06 0.0000000000e+00)
(1.0000000000e-01 7.4274379000e-06 0.0000000000e+00)
)
            \end{lstlisting}
            Velocity field file \texttt{U} is very similar in structure:
            \begin{lstlisting}[language=C++]
4 //number of points
(
(0.0000000000e+00 0.0000000000e+00 0.0000000000e+00) //vector components
(1.5774773000e+00 2.9655442000e-06 0.0000000000e+00)
(3.5008039000e+00 1.4358657000e-05 0.0000000000e+00)
(5.8499814000e+00 4.0020820000e-05 0.0000000000e+00)
)
            \end{lstlisting}

        \subsection{Pressure boundary conditions}
            A careful reader might have noticed that \oFoam is somewhat contradictory in one place. Providing boundary conditions for both pressure and velocity at the inlet to the domain is necessary.
            If those boundary conditions are not exactly complementary, it can be seen as overdetermining the Navier-Stokes equations.

            In the first of described cases this problem can be easily avoided since both zero Neumann condition for pressure and uniform velocity value are correct boundary conditions. Unfortunately neither \textbf{A-B}, nor \textbf{B-C} boundaries are defined along streamlines, and velocity field is not uniform in their vicinity, which means that zero Neumann boundary condition for pressure cannot be applied here.
            Instead \texttt{fixedFluxExtrapolatedPressure} type of boundary condition should be chosen. It is described in the \oFoam documentation as:
            \say{This boundary condition sets the pressure gradient to the provided value such that the flux on the boundary is that specified by the velocity boundary condition.}
