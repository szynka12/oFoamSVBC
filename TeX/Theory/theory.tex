\chapter{Theoretical background}
    \section{Turbulence and RANS equations}
        \label{ch::turbulence}
        Each flow can be characterised by a non dimensional number called the \emph{Reynolds number}:
        \begin{equation}
        Re = \dfrac{uL}{\nu}
        \end{equation}
        In the above expression symbols $u$, $L$ and $\nu$ denotes:
        \begin{itemize}
        \item[$u$] - velocity scale of the flow (e.g inlet velocity);
        \item[$L$] - characteristic length of the domain (e.g height of the step in backward facing step case);
        \item[$\nu$] - kinematic viscosity of the fluid.
        \end{itemize}
        Experiments conducted by Osborne Reynolds~\cite{Reynolds1883} prove that for low Reynolds numbers flow will remain stable even under the influence of big disturbances (e.g. separation, wall surface roughness).
        However, above certain Reynolds number each flow will lose it's laminar nature and become unstable even under the influence of very small disturbances. 
        
        Resulting oscillating flow is called turbulent. It can be described by three distinct features:
        \begin{itemize}
            \item presence of eddies and swirling motion;
            \item chaotic movement of the flow (small changes in initial conditions will cause drastic changes in flow appearance);
            \item greatly increased diffusion of any flow quantity by turbulent oscillations. 
        \end{itemize}

        Turbulence can be simulated directly. 
        Navier-Stokes equations fully describes flow of any fluid, turbulent motion included.
        Such numerical experiments are named \emph{DNS} (Direct Numerical Simulation).
        Unfortunately such approach has many disadvantages:
        \begin{itemize}
        \item in order to simulate any turbulent flow very fine mesh is necessary~\cite{Wilcox}
        \item simulation of that kind can not be simplified to a two dimensional case - turbulence is inherently a three dimensional phenomenon~\cite{landau2013fluid};
        \item DNS simulation generates a lot of data and full simulation may take days or months.
        \end{itemize}

        As of today DNS is not a viable option option for industrial CFD applications~\cite{Spalart}. Luckily there are other means to describe effects that turbulence has on fluid flow.
        In each turbulent flow two components of movement can be easily distinguished - one describing mean motion of he flow and the other corresponding to turbulent oscillations and eddies. 
        Given that, any field $\phi$ describing such flow can be decomposed into two separate fields:
        \begin{equation}
        \phi = \rans{\phi} + \phi'
        \end{equation}
        In above expression $\rans{\phi}$ denotes time averaged value~\cite{Wilcox} of $\phi$ and $\phi'$ describes turbulent fluctuations (see fig \ref{fig:RANSdecom}).

        \begin{figure}
            \centering
            \includestandalone[width = \linewidth]{Theory/tikz/rans}
            \label{fig:RANSdecom}
            \caption{Decomposition of field variable $\phi$ into time average and fluctuation part}
        \end{figure}

        In general data describing fluctuations $\phi'$ is not useful. Mean properties of the flow, on the other hand, provide much more advantageous information.
        In two similar flows, instantaneous image of two turbulent flows will be drastically different as fluctuations are chaotic in nature.
        Mean values of e.g. velocity fields will be comparable.

        In order to achieve quantitative description of fluid flow in terms of averaged variables $\rans{\bVec{u}}$ and $\rans{p}$ two operations on Navier-Stokes equations must be made:
        \begin{enumerate}
        \item Each variable must be decomposed into it's averaged and oscillating parts.
        \item Whole system of Navier-Stokes equations must be averaged in time.
        \end{enumerate}
        Having performed them, several of arising terms can be dropped and simplified resulting in \emph{Reynolds Averaged Navier-Stokes} (RANS) equations:
        \begin{equation}
        \begin{cases}
            &\pderiv{\rans{\bVec{u}}}{t} + (\rans{\bVec{u}} \cdot \nabla)\rans{\bVec{u}} = -\nabla \Big(\dfrac{\rans{p}}{\rho} \Big) %
            + \nabla\cdot \Big(\nu\nabla\rans{\bVec{u}} + \nu\nabla^T \rans{\bVec{u}} + \bm{\tau} \Big)\\
            &\nabla \cdot \rans{\bVec{u}} = 0
        \end{cases}
        \end{equation}
        The tensor $\bm{\tau}$ is called Reynolds stress tensor. It is symmetric and its components are defined in the following way:
        \begin{equation}
        \tau_{ij} = -\rans{u_i' u_j'}
        \end{equation}

        It represents effect turbulence has on the flow. Observing the above equation two remarks can be made.
        Firstly $\bm{\tau}$ is present in the exact part of equation that is "responsible" for viscous dissipation i.e. diffusion of momentum.
        It corresponds with one of the aforementioned features of turbulence: eddies increase mixing of the velocity field.
        Secondly, if Reynolds tensor is dropped from the equation original Navier-Stokes formulation is recovered.

        Unfortunately components of $\bm{\tau}$ are dependent on fluctuating parts of velocity field.
        Since no assumptions can be made about turbulent fluctuations, values of all six components of this tensor must be modelled.
        Te process of modelling components of $\bm{\tau}$ is often denoted as "closure".

        First step in closing RANS equation is assuming that Boussinesq eddy viscosity theory holds.
        It states that momentum transfer caused by turbulent eddies can be modelled by adding artificial eddy viscosity.
        Tensor $\bm{\tau}$ is then defined as:
        \begin{equation}
            \tau_{ij} =  2\nu_T S_{ij} - \dfrac{2}{3}\rans{u_\alpha' u_\alpha'}\delta_{ij} = 2\nu_T \Big(\pderiv{\rans{u}_i}{x_j} + \pderiv{\rans{u}_j}{x_i} \Big) - \dfrac{2}{3}\rans{u_\alpha' u_\alpha'}\delta_{ij}
        \end{equation}
        The last term on the right can be ignored for non-supersonic speed flows~\cite{Wilcox}.
        The newly introduced viscosity must be calculated by a chosen turbulent model.
        RANS equation after substituting Boussinesq assumption looks as follows:
        \begin{equation}
            \begin{cases}
            &\pderiv{\rans{u}_i}{t} + \rans{u}_j \pderiv{\rans{u}_i}{x_j} = -\pderiv{}{x_i}\Big(\dfrac{\rans{p}}{\rho}\Big) %
            + \pderiv{}{x_j}\Big[ (\nu + \nu_T)\Big( \pderiv{\rans{u}_i}{x_j} + \pderiv{\rans{{u}_j}}{{x_i}} \Big) \Big]\\
            &\pderiv{\rans{u}_k}{x_j} = 0
            \end{cases}
        \end{equation}

    \section{Wilcox k-$\omega$ turbulence model}
        
        In order to calculate turbulent viscosity $\nu_T$ system of two partial differential equations is introduced
        \begin{equation}
            \begin{cases}
            &\pderiv{k}{t} + \rans{u}_j \pderiv{k}{x_j} = P_k - \beta^*\omega k + %
            \pderiv{}{x_j}\Big[ (\nu + \sigma_k \dfrac{k}{\omega}) \pderiv{k}{x_j} \Big]\\
            &\pderiv{\omega}{t} + \rans{u}_j \pderiv{\omega}{x_j} = \dfrac{\gamma \omega}{k}P_k - \beta\omega^2 + %
            \pderiv{}{x_j}\Big[ (\nu + \sigma_\omega \dfrac{k}{\omega}) \pderiv{\omega}{x_j} \Big]
            \end{cases}
        \end{equation}
        First one is the equation for transport of kinetic energy of turbulent fluctuations $k$, that is defined in a following way:
        \begin{equation}
            k = \dfrac{1}{2}\rans{u'_i u'_i}
            \label{eq:k_def}
        \end{equation}
        Another scalar variable that is solved for is $\omega$ - the rate at which turbulence kinetic energy is dissipated per unit volume and time. Sometimes it is also referred to as the mean frequency of the turbulence. 
        Using those two quantities $\nu_T$ can be computed:
        \begin{equation}
            \nu_T = \dfrac{k}{\omega}
        \end{equation}


        First term on the right in the first equation is denoted as production of kinetic energy and calculated with following formula:
        \begin{equation}
            P_k = \nu_T S^2,\quad S = \sqrt{2S_{ij}S_{ij} }
        \end{equation}
        Constants present in the model have following values:
        \begin{equation*}
            \sigma_k = \sigma_\omega = 0.5,\quad \gamma = \dfrac{5}{9}
        \end{equation*}

        \begin{equation*}
            \beta = \dfrac{3}{40},\quad \beta^* = 0.09
        \end{equation*}

        In order to solve this system of equations appropriate boundary conditions must be specified. Boundary condition for $k$ can be easily deduced from it's definition (\ref{eq:k_def}):
        \begin{equation}
            k_{wall} = 0
        \end{equation}
        The same boundary condition for $\omega$ can be obtained by asymptotic analysis of the model behaviour near the solid wall~\cite{Wilcox}:
        \begin{equation}
            \omega_{wall} = \dfrac{6\nu}{\beta d_1^2}
        \end{equation}
        Variable $d_1$ is the distance from solid wall to the nearest integration point.
        Additionally inlet boundary conditions must be specified; for $k$:
        \begin{equation*}
            k_{inlet} = \dfrac{3}{2}(\vMag{\bVec{u}}t_i)^2
        \end{equation*}
        and for $\omega$:
        \begin{equation*}
            \omega{inlet} = \dfrac{\sqrt{k_{inlet}}}{l_T}
        \end{equation*}

        
